%presentation
\documentclass{beamer}

%impressions
%\documentclass[handout]{beamer}
%\usepackage{pgfpages}
%\pgfpagesuselayout{2 on 1}[a4paper,border shrink=5mm]
%\setbeameroption{notes on second screen}
%\pgfpagelayout{2 on 1}[a4paper, border, shrink=5mm]
% vue sur http://wwwtaketorg/spip/articlephp3?id_article=30...
\usepackage[T1]{fontenc}
\usepackage[frenchb]{babel}
\usepackage[utf8x]{inputenc} % Pour pouvoir taper les accents sans faire de combinaison
%\usepackage{arev}
%\usepackage{aeguill}
%mode code
\usepackage{listings}

%mode verbatim
\usepackage{moreverb}

%\usepackage[darktab]{beamerthemesidebar}
%\leftsidebar
%\usetheme{Hannover}
%\usetheme{Warsaw}
%\usetheme{PaloAlto}
\usetheme{JuanLesPins}
%\usetheme{Antibes}
%\usetheme{Shingara}
%\usetheme{Berlin}
%\usetheme{Oxygen}
\usepackage{thumbpdf}
\usepackage{wasysym}
\usepackage{ucs}
\usepackage{pgfarrows,pgfnodes,pgfautomata,pgfheaps,pgfshade}
\usepackage{verbatim}
\usepackage{color}

\AtBeginSection[]
{
  \begin{frame}<beamer>
    \tableofcontents[currentsection,currentsubsection]
  \end{frame}
}


\title{Merb, Le framework tellement bien qu'il sera intégré dans Rails}
\author{Cyril Mougel}

\logo{\includegraphics[width=10mm]{merb.png}}

\begin{document}

\begin{frame}
    \titlepage
\end{frame}

\section{Merb, c'est quoi ?}

\begin{frame}
	\frametitle{Un coquille vide ?}
	\begin{itemize}
		\item Non, car utilisé dans plein de projet
		\item Non, car sinon il ne serait pas mergé dans Rails
		\item Non, car il a entrainé des flameware comme Vim/Emacs
        \item Non, car la majeur partie des Merbistes sont des Railers
	\end{itemize}
\end{frame}

\begin{frame}
    \frametitle{C'est partie comment ?}
    \begin{itemize}
        \item "Start has a hack"
        \item http://pastie.org/14416
        \item Ezmobius et Wycats
        \item "No code is faster than no code"
    \end{itemize}
\end{frame}

\section{Les concepts de Merb}

\begin{frame}
    \frametitle{La différence par rapport à Ruby On Rails ?}
    \begin{itemize}
        \item ORM Agnostic
        \item Javascript Agnostic
        \item Modulaire
        \item API Public
    \end{itemize}
\end{frame}

\begin{frame}
    \frametitle{Les points communs avec Ruby On Rails}
    \begin{itemize}
        \item MVC
        \item View Agnostic
        \item Rack based (nouveau chez Rails 2.2)
        \item Thread safe (nouveau chez Rails 2.2)
    \end{itemize}
\end{frame}

\section{Les différentes partie de Merb}

\begin{frame}
    \frametitle{Merb-core}
    \begin{itemize}
        \item équivalent à ActionPack mais en light
        \item Routing, Rack, Bootloader, Controller
    \end{itemize}
\end{frame}

\begin{frame}
    \frametitle{Merb-more}
    \begin{itemize}
        \item Le reste d'ActionPack
        \item Des gems pratiques pour certaine options (merb-mailer,
                merb-cache, merb-action-args)
        \item Utile, mais non critique
    \end{itemize}
\end{frame}

\begin{frame}
    \frametitle{Merb-plugins}
    \begin{itemize}
        \item Maintenu par le Merb Core Team
        \item Les gems de communication avec les différentes briques
        (merb-sequel, merb-datamapper, merb-activerecord, etc...)
        \item Non essentiel
    \end{itemize}
\end{frame}

\section{En pratique ?}

\begin{frame}
    \frametitle{Render - Display - provides}
    \begin{itemize}
        \item On affiche une vue (Render)
        \item On rend une resource (Display)
        \item On fournit des formats (provides)
    \end{itemize}
\end{frame}

\begin{frame}
    \frametitle{Render}

    Un exemple de controller utilisant render :


    \begin{center}
        \lstinputlisting[language=Ruby,basicstyle=\scriptsize,
        numbers=left]{controller_with_render.rb}
    \end{center}
\end{frame}

\begin{frame}
    \frametitle{Display}

    Un exemple de controller utilisant Display :


    \begin{center}
        \lstinputlisting[language=Ruby,basicstyle=\scriptsize,
        numbers=left]{controller_with_display.rb}
    \end{center}
\end{frame}

\end{document}
