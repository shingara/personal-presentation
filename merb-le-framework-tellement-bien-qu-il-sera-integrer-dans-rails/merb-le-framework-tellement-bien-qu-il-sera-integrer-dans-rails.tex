%presentation
\documentclass{beamer}

%impressions
%\documentclass[handout]{beamer}
%\usepackage{pgfpages}
%\pgfpagesuselayout{2 on 1}[a4paper,border shrink=5mm]
%\setbeameroption{notes on second screen}
%\pgfpagelayout{2 on 1}[a4paper, border, shrink=5mm]
% vue sur http://wwwtaketorg/spip/articlephp3?id_article=30...
%\usepackage[T1]{fontenc}
\usepackage[frenchb]{babel}
\usepackage[utf8x]{inputenc} % Pour pouvoir taper les accents sans faire de combinaison
%\usepackage{arev}
%\usepackage{aeguill}
%mode code
\usepackage{listings}

%mode verbatim
\usepackage{moreverb}

%\usepackage[darktab]{beamerthemesidebar}
%\leftsidebar
%\usetheme{Hannover}
%\usetheme{Warsaw}
%\usetheme{PaloAlto}
\usetheme{JuanLesPins}
%\usetheme{Antibes}
%\usetheme{Shingara}
%\usetheme{Berlin}
%\usetheme{Oxygen}
\usepackage{thumbpdf}
\usepackage{wasysym}
\usepackage{ucs}
\usepackage{pgfarrows,pgfnodes,pgfautomata,pgfheaps,pgfshade}
\usepackage{verbatim}
\usepackage{color}

\AtBeginSection[]
{
  \begin{frame}<beamer>
    \tableofcontents[currentsection,currentsubsection]
  \end{frame}
}


\title{Merb, Le framework tellement bien qu'il sera intégré dans Rails}
\author{Cyril Mougel}
\date{07 Mars 2009}

\logo{\includegraphics[width=10mm]{merb.png}}

\begin{document}

\begin{frame}
    \titlepage
\end{frame}

\section{Merb, c'est quoi ?}

\begin{frame}
	\frametitle{Un coquille vide ?}
	\begin{itemize}
		\item Non, car utilisé dans plein de projets
		\item Non, car sinon il ne serait pas mergé dans Rails
		\item Non, car il a entrainé des flamewars comme Vim/Emacs
        \item Non, car la majeure partie des Merbistes sont des Railers
	\end{itemize}
\end{frame}

\begin{frame}
    \frametitle{C'est parti comment ?}
    \begin{itemize}
        \item "Started as a hack"
        \item http://pastie.org/14416
        \item Ezmobius et Wycats
        \item "No code is faster than no code"
    \end{itemize}
\end{frame}

\section{Les concepts de Merb}

\begin{frame}
    \frametitle{La différence par rapport à Ruby On Rails ?}
    \begin{itemize}
        \item ORM-agnostic
        \item Javascript-agnostic
        \item Modulaire
        \item API publique
    \end{itemize}
\end{frame}

\begin{frame}
    \frametitle{Les points communs avec Ruby On Rails}
    \begin{itemize}
        \item MVC
        \item View-agnostic
        \item Rack-based (nouveau depuis Rails 2.2)
        \item Thread-safe (nouveau depuis Rails 2.2)
    \end{itemize}
\end{frame}

\section{Les différentes parties de Merb}

\begin{frame}
    \frametitle{Merb-core}
    \begin{itemize}
        \item Équivalent à ActionPack mais en light
        \item Routing, Rack, Bootloader, Controller
    \end{itemize}
\end{frame}

\begin{frame}
    \frametitle{Merb-more}
    \begin{itemize}
        \item Le reste d'ActionPack
        \item Des gems pratiques pour certaines options (merb-mailer,
                merb-cache, merb-action-args)
        \item Utile, mais non critique
    \end{itemize}
\end{frame}

\begin{frame}
    \frametitle{Merb-plugins}
    \begin{itemize}
        \item Maintenu par la Merb Core Team
        \item Les gems de communication avec les différentes briques
        (merb-sequel, merb-datamapper, merb-activerecord, etc...)
        \item Non essentiel
    \end{itemize}
\end{frame}

\section{En pratique ?}

\begin{frame}
    \frametitle{render - display - provides}
    \begin{itemize}
        \item On affiche une vue (\texttt{render})
        \item On rend une resource (\texttt{display})
        \item On fournit des formats (\texttt{provides})
    \end{itemize}
\end{frame}

\begin{frame}
    \frametitle{Render}

    Un exemple de controller utilisant render :

    \begin{center}
        \lstinputlisting[language=Ruby,basicstyle=\scriptsize, numbers=left]{controller_with_render.rb}
    \end{center}
\end{frame}

\begin{frame}
    \frametitle{Display}

    Un exemple de controller utilisant Display :

    \begin{center}
        \lstinputlisting[language=Ruby,basicstyle=\scriptsize,numbers=left]{controller_with_display.rb}
    \end{center}
\end{frame}

\begin{frame}
    \frametitle{merb-action-args}

    \begin{itemize}
        \item Des paramètres sur les actions ?
        \item Simplifie les controllers
        \item Basé sur ParseTree
    \end{itemize}
\end{frame}

\begin{frame}
    Sans
    \\
            \lstinputlisting[language=Ruby,numbers=left,basicstyle=\tiny]{controller_without_actions-args.rb}

    Avec
    \\

            \lstinputlisting[language=Ruby,numbers=left,basicstyle=\tiny]{controller_with_actions-args.rb}
\end{frame}

\begin{frame}
  \frametitle{Les routeurs}

  \begin{itemize}
    \item Ressources ( \texttt{resources :articles} )
    \item Très lisible ( 
      \texttt{match("/about").to(:controller => "main", :action => "about").name(:about)} )
    \item \texttt{url(:about)}
  \end{itemize}
\end{frame}

\begin{frame}
    \frametitle{merb-slice}
    \begin{itemize}
        \item Inspiré des apps de Django
        \item Une sorte de mini application Merb
        \item On y définit :
        \begin{itemize}
            \item Ses routes
            \item Ses controlleurs dans un namespace
            \item Ses modèles
        \end{itemize}
        \item router.rb : \texttt{slice(:merb\_static\_pages\_slice, :path => "static")}
        \item merb-auth: la slice d'authentification
    \end{itemize}
\end{frame}

\begin{frame}
    \frametitle{merb-cache}
    \begin{itemize}
        \item Une seule méthode : \texttt cache
        \item On invalide le cache avec \texttt{eager\_cache}
        \item Pas de différentes méthodes, juste de différentes options
    \end{itemize}
\end{frame}

\begin{frame}
    \begin{center}
        \lstinputlisting[language=Ruby,basicstyle=\tiny, numbers=left]{controller_with_cache.rb}
    \end{center}
\end{frame}

\begin{frame}
    \frametitle{Les ressources ?}
    \begin{itemize}
        \item http://wiki.merbivore.org
        \item http://merbivore.org/documentation.html
        \item http://merbunity.com/
    \end{itemize}
\end{frame}

\begin{frame}
    \begin{center}
    \huge{}
    questions ?
    \end{center}
\end{frame}

\end{document}
