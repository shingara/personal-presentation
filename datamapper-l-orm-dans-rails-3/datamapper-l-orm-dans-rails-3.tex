%presentation
\documentclass{beamer}

%impressions
%\documentclass[handout]{beamer}
%\usepackage{pgfpages}
%\pgfpagesuselayout{2 on 1}[a4paper,border shrink=5mm]
%\setbeameroption{notes on second screen}
%\pgfpagelayout{2 on 1}[a4paper, border, shrink=5mm]
% vue sur http://wwwtaketorg/spip/articlephp3?id_article=30...
%\usepackage[T1]{fontenc}
\usepackage[frenchb]{babel}
\usepackage[utf8x]{inputenc} % Pour pouvoir taper les accents sans faire de combinaison
%\usepackage{arev}
%\usepackage{aeguill}
%mode code
\usepackage{listings}

%mode verbatim
\usepackage{moreverb}

%\usepackage[darktab]{beamerthemesidebar}
%\leftsidebar
%\usetheme{Hannover}
%\usetheme{Warsaw}
%\usetheme{PaloAlto}
\usetheme{JuanLesPins}
%\usetheme{Antibes}
%\usetheme{Shingara}
%\usetheme{Berlin}
%\usetheme{Oxygen}
\usepackage{thumbpdf}
\usepackage{wasysym}
\usepackage{ucs}
\usepackage{pgfarrows,pgfnodes,pgfautomata,pgfheaps,pgfshade}
\usepackage{verbatim}
\usepackage{color}

\AtBeginSection[]
{
  \begin{frame}<beamer>
    \tableofcontents[currentsection,currentsubsection]
  \end{frame}
}


\title{DataMapper, l'ORM dans Rails 3 ?}
\author{Cyril Mougel}
\date{07 Mars 2009}

\logo{\includegraphics[width=30mm]{datamapper.jpeg}}

\lstset{ breaklines=true,language=ruby,numbers=left,tabsize=2
    , basicstyle=\small\ttfamily
    , keywordstyle=\color{blue}
    , commentstyle=\color{green}
    , stringstyle=\color{red}
    , identifierstyle=\ttfamily
    , columns=fixed
}

\begin{document}

\begin{frame}
    \titlepage
\end{frame}

\section{DataMapper c'est quoi ?}

\begin{frame}
	\frametitle{DataMapper c'est quoi ?}
	\begin{itemize}
        \item Commencé en Février 2007
        \item Objet Relationel Model (Model Object Relationnel)
        \item Supporté uniquement par la communauté
	\end{itemize}
\end{frame}

\begin{frame}
    \frametitle{Pourquoi DataMapper ?}
    \begin{itemize}
        \item L'identity Map
        \item La gestion des multi-repository natif
        \item Lazy loading
        \item Eager loading
        \item Tout est ruby
        \item Modulaire
    \end{itemize}
\end{frame}

\begin{frame}
    \frametitle{Identity Map}
    \begin{itemize}
        \item \lstinline{User.first == User.first}
        \item \lstinline{User.first == User.get(1)}
    \end{itemize}
\end{frame}

\begin{frame}
    \frametitle{Gestion des multi-repository natif}
    \begin{itemize}
        \item \lstinline{User.all(:repository => :other)}
        \item \lstinline{User.all \# :repository => :default}
    \end{itemize}
\end{frame}

\begin{frame}
    \frametitle{Lazy Loading}
    \begin{center}
        \lstinputlisting[basicstyle=\scriptsize]{lazy_loading.rb}
    \end{center}
\end{frame}

\begin{frame}
    \frametitle{Eager Loading}
    \begin{center}
        \lstinputlisting[basicstyle=\scriptsize]{eager_loading.rb}
    \end{center}
\end{frame}

\section{Les base de DataMapper}

\begin{frame}
    \frametitle{Un classe Model}
    \begin{itemize}
        \item Hérite seulement de Object
        \item Include de \lstinline{include DataMapper::Resource}
        \item Les propriétés déclaré dans les classes modèles
    \end{itemize}
\end{frame}

\begin{frame}
    \begin{center}
        \lstinputlisting{model_user.rb}
    \end{center}
\end{frame}

\begin{frame}
    \frametitle{La recherche avec DataMapper}
    \begin{center}
        \lstinputlisting{search_user.rb}
    \end{center}
\end{frame}

\section{Les plugins DataMapper}

\begin{frame}
    \frametitle{dm-validations}

    \begin{itemize}
        \item validates\_xxx
        \item validation dans les properties directement
    \end{itemize}

    \begin{center}
        \lstinputlisting[basicstyle=\scriptsize]{validate_user.rb}
    \end{center}
\end{frame}

\begin{frame}
    \frametitle{dm-migrations}

    \begin{center}
        \lstinputlisting[basicstyle=\scriptsize]{user_migration.rb}
    \end{center}
\end{frame}

\begin{frame}
    \frametitle{plein d'autre plugins}

    \begin{itemize}
        \item dm-types
        \item dm-timestamp
        \item dm-tags
        \item dm-sweatshop
        \item dm-constraints
        \item dm-is-list
        \item etc..
    \end{itemize}

\end{frame}

\begin{frame}
    \frametitle{Les ressources ?}
    \begin{itemize}
        \item http://datamapper.org
        \item http://github.com/datamapper
        \item http://groups.google.com/group/datamapper
    \end{itemize}
\end{frame}

\begin{frame}
    \begin{center}
    \huge{}
    questions ?
    \end{center}
\end{frame}

\end{document}
