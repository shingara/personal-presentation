%presentation
\documentclass{beamer}

%impressions
%\documentclass[handout]{beamer}
%\usepackage{pgfpages}
%\pgfpagesuselayout{2 on 1}[a4paper,border shrink=5mm]
%\setbeameroption{notes on second screen}
%\pgfpagelayout{2 on 1}[a4paper, border, shrink=5mm]
% vue sur http://wwwtaketorg/spip/articlephp3?id_article=30...
\usepackage[T1]{fontenc}
\usepackage[frenchb]{babel}
\usepackage[utf8x]{inputenc} % Pour pouvoir taper les accents sans faire de combinaison
%\usepackage{arev}
%\usepackage{aeguill}
%mode code
\usepackage{listings}

%mode verbatim
\usepackage{moreverb}

%\usepackage[darktab]{beamerthemesidebar}
%\leftsidebar
%\usetheme{Hannover}
%\usetheme{Warsaw}
%\usetheme{PaloAlto}
\usetheme{JuanLesPins}
%\usetheme{Antibes}
%\usetheme{Shingara}
%\usetheme{Berlin}
%\usetheme{Oxygen}
\usepackage{thumbpdf}
\usepackage{wasysym}
\usepackage{ucs}
\usepackage{pgfarrows,pgfnodes,pgfautomata,pgfheaps,pgfshade}
\usepackage{verbatim}
\usepackage{color}

\AtBeginSection[]
{
  \begin{frame}<beamer>
    \tableofcontents[currentsection,currentsubsection]
  \end{frame}
}


\title{Ruby On Rails par Rspec}
\author{Cyril Mougel}

\logo{\includegraphics[width=10mm]{rails.png}}

\begin{document}

\begin{frame}
    \titlepage
\end{frame}

\section{Présentation}

\begin{frame}
	\frametitle{Typo 5.1.3}
	\begin{itemize}
		\item supporte uniquement Rails 2.0.2
		\item Pas d'évolution vers Rails 2.1
		\item Couverture de code
	\end{itemize}
\end{frame}

\section{Migration lié à Rails 2.1}

\begin{frame}
    \frametitle{Mise à jour du projet}
    \begin{itemize}
        \item rake rails:update
        \item Mise à jour des fichiers de boot
        \item Mise à jour des fichiers de JS
    \end{itemize}
\end{frame}

\begin{frame}
    \frametitle{Mais pourquoi Rspec alors ?}
    \begin{itemize}
        \item Rspec == BDD Framework
        \item Documentation g\'en\'er\'ee plus claire que les Test::Unit
        \item R\'eutilisation plus simple du comportement
        \item Ajout des Stories (Cucumber)
    \end{itemize}
\end{frame}

\begin{frame}
    \frametitle{Et ca s'installe comment ?}
    \begin{itemize}
        \item gem install rspec \&\& gem install rspec-rails
        \item gem install rspec \&\& ./script/install git://github.com/dchelimsky/rspec-rails.git
        \item gem install rspec \&\& git-submodule add vendor/plugins/rspec-rails git://github.com/dchelimsky/rspec-rails.git
    \end{itemize}
\end{frame}

\section{La couche mod\`ele de Rails}

\begin{frame}
    \frametitle{Des classes de Mapping a la Base de donn\'ee}
    \begin{center}
        \lstinputlisting[language=Ruby,basicstyle=\scriptsize,
        numbers=left]{user_products.rb}
    \end{center}
\end{frame}

\begin{frame}
    \frametitle{Doit cr\'eer les accesseurs sur les colonnes}
    \begin{center}
        \lstinputlisting[language=Ruby,basicstyle=\scriptsize,
        numbers=left]{user_accessor.rb}
    \end{center}
\end{frame}

\begin{frame}
    \frametitle{Doit valider le model}
    \begin{center}
        \lstinputlisting[language=Ruby,basicstyle=\tiny,
        numbers=left]{user_valid_model.rb}
    \end{center}
\end{frame}

\begin{frame}
    \frametitle{Doit chercher des donn\'ees}
    \begin{center}
        \lstinputlisting[language=Ruby,basicstyle=\tiny,
        numbers=left]{user_find.rb}
    \end{center}
\end{frame}

\begin{frame}
    \frametitle{Doit avoir des associations}
    \begin{center}
        \lstinputlisting[language=Ruby,basicstyle=\tiny,
        numbers=left]{user_association.rb}
    \end{center}
\end{frame}

\section{La couche controller de Rails}

\begin{frame}
    \frametitle{Un controlleur Rails}
    \begin{center}
        \lstinputlisting[language=Ruby,basicstyle=\tiny,
        numbers=left]{user_controller_show.rb}
    \end{center}
\end{frame}

\begin{frame}
    \frametitle{Doit permettre de voir la liste des utilisateurs}
    \begin{center}
        \lstinputlisting[language=Ruby,basicstyle=\tiny,
        numbers=left]{user_controller_index_spec.rb}
    \end{center}
\end{frame}

\begin{frame}
    \frametitle{Doit permettre de voir un utilisateur particulier}
    \begin{center}
        \lstinputlisting[language=Ruby,basicstyle=\tiny,
        numbers=left]{user_controller_show_spec.rb}
    \end{center}
\end{frame}

\begin{frame}
    \frametitle{Doit cr\'eer un utilisateur}
    \begin{center}
        \lstinputlisting[language=Ruby,basicstyle=\tiny,
        numbers=left]{user_controller_create_spec.rb}
    \end{center}
\end{frame}

\section {Les routes Rails}

\begin{frame}
    \frametitle{Une simple ligne de route}
    \begin{center}
        \lstinputlisting[language=Ruby,basicstyle=\tiny,
        numbers=left]{user_route.rb}
    \end{center}
\end{frame}

\begin{frame}
    \frametitle{Doit cr\'eer plein de routes}
    \begin{center}
        \lstinputlisting[language=Ruby,basicstyle=\tiny,
        numbers=left]{user_route_spec.rb}
    \end{center}
\end{frame}

\section {La couche Vue de Rails}

\begin{frame}
    \frametitle{Une vue d'index}
    \begin{center}
        \lstinputlisting[language=Ruby,basicstyle=\tiny,
        numbers=left]{index.html.erb}
    \end{center}
\end{frame}

\begin{frame}
    \frametitle{Doit permettre de voir les utilisateurs}
    \begin{center}
        \lstinputlisting[language=Ruby,basicstyle=\tiny,
        numbers=left]{index.html.erb_spec.rb}
    \end{center}
\end{frame}

\section{les mocks ?}

\begin{frame}
  \frametitle{S'il te plait, dessine moi un mock ?}
  \begin{itemize}
    \item mock\_model(User)
    \item Comportement d'un objet ActiveRecord sans access a la BDD
    \item Possibilite de retourner ce que l'on veux
    \item Evite de creer une fixture qui gere ce cas la
  \end{itemize}
\end{frame}

\begin{frame}
  \frametitle{Doit s'utiliser dans les controllers}
  \begin{center}
    \lstinputlisting[language=Ruby,basicstyle=\tiny,
      numbers=left]{user_controller_mock.rb}
  \end{center}
\end{frame}


\end{document}
