%presentation
\documentclass{beamer}

%impressions
%\documentclass[handout]{beamer}
%\usepackage{pgfpages}
%\pgfpagesuselayout{2 on 1}[a4paper,border shrink=5mm]
%\setbeameroption{notes on second screen}
%\pgfpagelayout{2 on 1}[a4paper, border, shrink=5mm]
% vue sur http://wwwtaketorg/spip/articlephp3?id_article=30...
\usepackage[frenchb]{babel}
\usepackage[utf8x]{inputenc} % Pour pouvoir taper les accents sans faire de combinaison
%\usepackage{arev}
%\usepackage{aeguill}
%mode code
\usepackage{listings}

%mode verbatim
\usepackage{moreverb}

%\usepackage[darktab]{beamerthemesidebar}
%\leftsidebar
%\usetheme{Hannover}
%\usetheme{Warsaw}
%\usetheme{PaloAlto}
\usetheme{JuanLesPins}
%\usetheme{Antibes}
%\usetheme{Shingara}
%\usetheme{Berlin}
%\usetheme{Oxygen}
\usepackage{thumbpdf}
\usepackage{wasysym}
\usepackage{ucs}
\usepackage{pgfarrows,pgfnodes,pgfautomata,pgfheaps,pgfshade}
\usepackage{verbatim}
\usepackage{color}

\AtBeginSection[]
{
  \begin{frame}<beamer>
    \tableofcontents[currentsection,currentsubsection]
  \end{frame}
}


\title{Migration de Rail 2.0 à 2.2}
\author{Cyril Mougel}

\logo{\includegraphics[width=10mm]{rails.png}}

\lstnewenvironment{ruby}{\lstset{language=ruby}}{}

\begin{document}

\begin{frame}
    \titlepage
\end{frame}

\section{Context}

\begin{frame}
	\frametitle{Typo 5.1.3}
	\begin{itemize}
		\item supporte uniquement Rails 2.0.2
		\item Pas d'évolution vers Rails 2.1
		\item Couverture de code
	\end{itemize}
\end{frame}

\section{Blocage lié à Rails 2.1}

\begin{frame}
    \frametitle{Mise à jour du projet}
    \begin{itemize}
        \item rake rails:update
        \item Mise à jour des fichiers de boot
        \item Mise à jour des fichiers de JS
    \end{itemize}
\end{frame}

\begin{frame}
    \frametitle{Plus de follow\_redirect dans les test de controlleur}
    \begin{itemize}
        \item Impossibilité d'utiliser \lstinline!follow_redirect! dans un
        \lstinline!Test::ActionController!
        \item Utilisation uniquement dans les test d'intégration
        \item Inutile dans les test d'action
    \end{itemize}
\end{frame}

\begin{frame}
    \frametitle{Ce qui est déprécié ou supprimé}
    \lstinline!render_partial! est supprimé suite à sa déprécation sur Rails
        2.0
        
    Utiliser \lstinline!render :partial!
\end{frame}

\section{Amélioration possible avec Rails 2.1}

\begin{frame}[fragile]
    \frametitle{Ajout des .last et .first}
    Plus besoin d'écrire find(:first) ou find(:last) 

    Maintenant User.first et User.last fonctionne.

    \begin{columns}
        \begin{column}[l]{4cm}
            \begin{lstlisting}[language=Ruby]
User.find(:first)
User.find(:last)
            \end{lstlisting}
        \end{column}
        \begin{column}[c]{1cm}
        =>
        \end{column}
        \begin{column}[r]{4cm}
            \begin{lstlisting}[language=Ruby]
User.first
User.last
            \end{lstlisting}
        \end{column}
    \end{columns}
\end{frame}

\begin{frame}[fragile]
    \frametitle{named\_scope}
            \begin{lstlisting}[language=Ruby,showstringspaces=false]
named_scope :published_articles, 
                :conditions => { :published => true }, 
                :order => 'published_at DESC'
            \end{lstlisting}
            \begin{lstlisting}[language=Ruby,showstringspaces=false]
named_scope :with_char, lambda { |*args| 
    :conditions => ['name LIKE ? ', 
                    "%#{args.first}%"]
}
            \end{lstlisting}
\end{frame}

\begin{frame}
    \frametitle{suivi des évolutions du model}
    \begin{itemize}
        \item article.changed?
        \item article.name\_changed?
        \item => UPDATE body='foo' FROM contents where id='12'
    \end{itemize}
\end{frame}

\section{Blocage lié à Rails 2.2}

\begin{frame}
  \frametitle{relative\_url\_root en configuration de ActionController}
  \begin{itemize}
    \item Le helper link\_to utilise relative\_url\_root
    \item Avant : utilisait @request.relative\_url\_root
    \item Maintenant : On défini directement dans la configuration :
    ActionController::Base.relative\_url\_root
  \end{itemize}
\end{frame}

\begin{frame}
    \frametitle{Migration dans des transactions}
    \begin{itemize}
        \item Si la migration échoue, pas de mise à jour de la BDD
        \item raise Exception == Migration Failed, même avec rescue
        \item Eviter tous les cas de Raise.
    \end{itemize}
\end{frame}

\begin{frame}
    \frametitle{Chargement des classes de controller avec cache\_classe = true}
    \begin{itemize}
        \item en production cache\_classe = true
        \item Chargement des classes en mémoire
        \item rake db:migrate RAILS\_ENV='production' charge classe 
        \item si appel ActiveRecord en cache du controller == FAILED
        \item rescue it
    \end{itemize}
\end{frame}

\begin{frame}
  \frametitle{et des trivialités}
  \begin{itemize}
    \item fragment\_cache\_store remplacer par cache\_store
    \item TextHelper::truncate prend un seul argument avec :length en option.
    La méthode avec plusieurs argument DEPRECATED
  \end{itemize}
\end{frame}

\section{Amélioration possible avec Rails 2.2}

\begin{frame}[fragile]
    \frametitle{les conditions par Hash}
    \begin{lstlisting}[Language=Ruby]
Article.all(:conditions => { 
        :created_at => 5.day.ago,
        :tags => { :name => 'foo' }})
    \end{lstlisting}
\end{frame}

\begin{frame}[fragile]
    \frametitle{facilité de mémoization}
    Utilisation simple de la mémoization
    \begin{columns}
        \begin{column}[l]{4cm}
            \begin{lstlisting}[language=Ruby, frame=single]
def fields
  @fields ||= []
end
            \end{lstlisting}
        \end{column}
        \begin{column}[c]{1cm}
         =>
        \end{column}
        \begin{column}[r]{4cm}
            \begin{lstlisting}[language=Ruby, frame=single]
def fields
  @fields = []
end
memoize :fields
            \end{lstlisting}
        \end{column}
    \end{columns}

\end{frame}

\begin{frame}
    \begin{center}
    \huge{}
    Question ?
    \end{center}
\end{frame}


\end{document}
