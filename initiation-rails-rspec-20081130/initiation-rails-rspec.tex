%presentation
\documentclass{beamer}

%impressions
%\documentclass[handout]{beamer}
%\usepackage{pgfpages}
%\pgfpagesuselayout{2 on 1}[a4paper,border shrink=5mm]
%\setbeameroption{notes on second screen}
%\pgfpagelayout{2 on 1}[a4paper, border, shrink=5mm]
% vue sur http://wwwtaketorg/spip/articlephp3?id_article=30...
\usepackage[T1]{fontenc}
\usepackage[frenchb]{babel}
\usepackage[utf8x]{inputenc} % Pour pouvoir taper les accents sans faire de combinaison
%\usepackage{arev}
%\usepackage{aeguill}
%mode code
\usepackage{listings}

%mode verbatim
\usepackage{moreverb}

%\usepackage[darktab]{beamerthemesidebar}
%\leftsidebar
%\usetheme{Hannover}
%\usetheme{Warsaw}
%\usetheme{PaloAlto}
\usetheme{JuanLesPins}
%\usetheme{Antibes}
%\usetheme{Shingara}
%\usetheme{Berlin}
%\usetheme{Oxygen}
\usepackage{thumbpdf}
\usepackage{wasysym}
\usepackage{ucs}
\usepackage{pgfarrows,pgfnodes,pgfautomata,pgfheaps,pgfshade}
\usepackage{verbatim}
\usepackage{color}

\AtBeginSection[]
{
  \begin{frame}<beamer>
    \tableofcontents[currentsection,currentsubsection]
  \end{frame}
}


\title{Ruby On Rails par Rspec}
\author{Cyril Mougel}

\logo{\includegraphics[width=10mm]{rails.png}}

\begin{document}

\begin{frame}
    \titlepage
\end{frame}

\section{Rspec, BDD ?}

\begin{frame}
	\frametitle{Behaviour Driven Development}
	\begin{itemize}
		\item M\'ethode Agile
		\item Extreme programming
		\item TDD (Test Driven Development)
        \item l'empire du 'should'
	\end{itemize}
\end{frame}

\section{Rails est agile avant tout}

\begin{frame}
    \frametitle{Test::Unit de base dans Rails}
    \begin{itemize}
        \item Dossier test cr\'e\'e \`a chaque cr\'eation de projet Rails
        \item Stats de nombre de LOC et LOC de test (TODO: mettre image ici)
        \item Possibilite de tester chaque couche de Rails
    \end{itemize}
\end{frame}

\begin{frame}
    \frametitle{Mais pourquoi Rspec alors ?}
    \begin{itemize}
        \item Rspec == BDD Framework
        \item Documentation g\'en\'er\'ee plus claire que les Test::Unit
        \item R\'eutilisation plus simple du comportement
        \item Ajout des Stories (Cucumber)
    \end{itemize}
\end{frame}

\section{La couche mod\`ele de Rails}

\begin{frame}
    \frametitle{Des classes de Mapping a la Base de donn\'ee}
    \begin{center}
        \lstinputlisting[language=Ruby,basicstyle=\scriptsize,
        numbers=left]{user_products.rb}
    \end{center}
\end{frame}

\begin{frame}
    \frametitle{Doit cr\'eer les accesseurs sur les colonnes}
    \begin{center}
        \lstinputlisting[language=Ruby,basicstyle=\scriptsize,
        numbers=left]{user_accessor.rb}
    \end{center}
\end{frame}

\begin{frame}
    \frametitle{Doit valider le model}
    \begin{center}
        \lstinputlisting[language=Ruby,basicstyle=\tiny,
        numbers=left]{user_valid_model.rb}
    \end{center}
\end{frame}

\begin{frame}
    \frametitle{Doit chercher des donn\'ees}
    \begin{center}
        \lstinputlisting[language=Ruby,basicstyle=\tiny,
        numbers=left]{user_find.rb}
    \end{center}
\end{frame}

\begin{frame}
    \frametitle{Doit avoir des associations}
    \begin{center}
        \lstinputlisting[language=Ruby,basicstyle=\tiny,
        numbers=left]{user_association.rb}
    \end{center}
\end{frame}

\end{document}
