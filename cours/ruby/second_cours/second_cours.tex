%presentation
\documentclass{beamer}

%impressions
%\documentclass[handout]{beamer}
%\usepackage{pgfpages}
%\pgfpagesuselayout{2 on 1}[a4paper,border shrink=5mm]
%\setbeameroption{notes on second screen}
%\pgfpagelayout{2 on 1}[a4paper, border, shrink=5mm]
% vue sur http://wwwtaketorg/spip/articlephp3?id_article=30...
%\usepackage[T1]{fontenc}
\usepackage[frenchb]{babel}
\usepackage[utf8x]{inputenc} % Pour pouvoir taper les accents sans faire de combinaison
%\usepackage{arev}
%\usepackage{aeguill}
%mode code
\usepackage{listings}

%mode verbatim
\usepackage{moreverb}

%\usepackage[darktab]{beamerthemesidebar}
%\leftsidebar
%\usetheme{Hannover}
%\usetheme{Warsaw}
%\usetheme{PaloAlto}
\usetheme{JuanLesPins}
%\usetheme{Antibes}
%\usetheme{Shingara}
%\usetheme{Berlin}
%\usetheme{Oxygen}
\usepackage{thumbpdf}
\usepackage{wasysym}
\usepackage{ucs}
\usepackage{pgfarrows,pgfnodes,pgfautomata,pgfheaps,pgfshade}
\usepackage{verbatim}
\usepackage{color}

\title{D\'ebuter en Ruby}
\author{Cyril Mougel}

\lstset{
  breaklines=true
    , language=ruby
    , numbers=left
    , tabsize=2
    , basicstyle=\small\ttfamily
    , keywordstyle=\color{blue}
    , commentstyle=\color{green}
    , stringstyle=\color{red}
    , identifierstyle=\ttfamily
    , columns=fixed
    , showstringspaces=false
}

\begin{document}

\begin{frame}
  \titlepage
\end{frame}

\begin{frame}
  \frametitle{require}
  \begin{itemize}
    \item inclus les fichiers dans la pile d'execution
    \item avec extension si path complet
    \item \$LOAD\_PATH et nom du fichier direct
  \end{itemize}
\end{frame}

\begin{frame}
  \begin{beamerboxesrounded}{require}
    \lstinputlisting[numbers=none,basicstyle=\tiny]{require.rb}
  \end{beamerboxesrounded}
\end{frame}

\begin{frame}
  \frametitle{if else}
  \begin{itemize}
    \item se ferme par un end
    \item elsif
  \end{itemize}
\end{frame}

\begin{frame}
  \begin{beamerboxesrounded}{if else elsif}
    \lstinputlisting[numbers=none,basicstyle=\tiny]{if_else.rb}
  \end{beamerboxesrounded}
\end{frame}

\begin{frame}
  \frametitle{test unitaire}
  \begin{itemize}
    \item H\'erite Test::Units::TestCase
    \item m\'ethode commençant par test\_
    \item assert\_equal
    \item assert
    \item http://www.ruby-doc.org/core/classes/Test/Unit.html
    \item setup
    \item teardown
  \end{itemize}
\end{frame}

\begin{frame}
  \begin{beamerboxesrounded}{Test Unitaire}
    \lstinputlisting[numbers=none,basicstyle=\tiny]{test_car_1.rb}
  \end{beamerboxesrounded}
\end{frame}

\begin{frame}
  \frametitle{Cr\'eation d'une classe}
  Ecrire une classe qui se nomme Car. Elle doit avoir une m\'ethode d'instance qui
  se nomme 'running?' et qui renvoi true ou false.
\end{frame}

\begin{frame}
  \begin{beamerboxesrounded}{Premier Exercice}
    \lstinputlisting[numbers=none,basicstyle=\tiny]{test_car_1.rb}
  \end{beamerboxesrounded}
\end{frame}

\begin{frame}
  La classe Car doit avoir une m\'ethode 'run' qui d\'emarre la voiture.
  Une fois la voiture d\'emarrer elle doit donc avoir la m\'ethode running? qui
  renvoi true
\end{frame}
\begin{frame}
  \begin{beamerboxesrounded}{Premier Exercice}
    \lstinputlisting[numbers=none,basicstyle=\tiny]{test_car_2.rb}
  \end{beamerboxesrounded}
\end{frame}

\begin{frame}
  Ajouter une m\'ethode 'stop' qui arrete la voiture de rouler. La m\'ethode running?
  doit donc renvoyer cette fois-ci false
\end{frame}
\begin{frame}
  \begin{beamerboxesrounded}{Premier Exercice}
    \lstinputlisting[numbers=none,basicstyle=\tiny]{test_car_3.rb}
  \end{beamerboxesrounded}
\end{frame}

\begin{frame}
  Indiquer un attribut d'instance accessible en \'ecriture qui indique le nombre de
  de fois que la voiture s'est arrêt\'ee.
\end{frame}
\begin{frame}
  \begin{beamerboxesrounded}{Premier Exercice}
    \lstinputlisting[numbers=none,basicstyle=\tiny]{test_car_4.rb}
  \end{beamerboxesrounded}
\end{frame}

\begin{frame}
  Permettre d'ajouter un nombre de d\'emarrage/Arrêt lors de la cr\'eation
  d'un objet 'Car'
\end{frame}
\begin{frame}
  \begin{beamerboxesrounded}{Premier Exercice}
    \lstinputlisting[numbers=none,basicstyle=\tiny]{test_car_5.rb}
  \end{beamerboxesrounded}
\end{frame}

\begin{frame}
  Ajouter une m\'ethode de classe 'nb\_car' qui indique le nombre de voiture actuellement en
  service.
\end{frame}
\begin{frame}
  \begin{beamerboxesrounded}{Premier Exercice}
    \lstinputlisting[numbers=none,basicstyle=\tiny]{test_car_6.rb}
  \end{beamerboxesrounded}
\end{frame}


\begin{frame}
  \frametitle{Mise en place d'H\'eritage}
  Avoir une classe Ford qui h\'erite de Car et ajoute une m\'ethode 'marque' qui renvoi l'information que c'est une Ford
  De même avec une Classe Renault
\end{frame}

\begin{frame}
  \begin{beamerboxesrounded}{H\'eritage}
    \lstinputlisting[numbers=none,basicstyle=\tiny]{test_car_7.rb}
  \end{beamerboxesrounded}
\end{frame}

\begin{frame}
  \frametitle{Mise en place de Mixin}
  Cr\'eer un module qui permette au class Ford et Renault d'avoir une m\'ethode 'add\_driver'
  qui ajoute des passagers à la voiture et une m\'ethode 'has\_driver?' qui renvoi true si il y a un driver.
\end{frame}
\begin{frame}
  \begin{beamerboxesrounded}{Mixin}
    \lstinputlisting[numbers=none,basicstyle=\tiny]{test_car_8.rb}
  \end{beamerboxesrounded}
\end{frame}

\end{document}
